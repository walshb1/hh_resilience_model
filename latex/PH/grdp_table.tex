\documentclass[10pt]{article}
\usepackage[usenames]{color} %used for font color
\usepackage{amssymb} %maths
\usepackage{amsmath} %maths
\usepackage[utf8]{inputenc} %useful to type directly diacritic characters
\usepackage{booktabs}
\begin{document}
\[\begin{tabular}{lrrr}
\toprule
{} &    AHI &     GRDP &  Ratio \\
region                   &         &          &        \\
\midrule
CAR                      &   102.8 &    234.6 &   43.8 \\
XIII - Caraga            &   111.5 &    159.0 &   70.1 \\
VI - Western Visayas     &   403.5 &    549.8 &   73.4 \\
V - Bicol                &   237.9 &    282.8 &   84.1 \\
ARMM                     &    98.6 &     99.6 &   99.0 \\
III - Central Luzon      &   699.3 &   1187.3 &   58.9 \\
II - Cagayan Valley      &   190.2 &    236.8 &   80.3 \\
IVA - CALABARZON         &   957.8 &   2059.5 &   46.5 \\
VII - Central Visayas    &   388.1 &    867.2 &   44.8 \\
X - Northern Mindanao    &   221.5 &    517.6 &   42.8 \\
XI - Davao               &   270.2 &    565.2 &   47.8 \\
VIII - Eastern Visayas   &   193.8 &    271.9 &   71.3 \\
I - Ilocos               &   293.4 &    409.1 &   71.7 \\
NCR                      &  1056.9 &   5043.6 &   21.0 \\
IVB - MIMAROPA           &   170.1 &    204.8 &   83.0 \\
XII - SOCCSKSARGEN       &   190.9 &    356.0 &   53.6 \\
IX - Zamboanga Peninsula &   169.6 &    277.2 &   61.2 \\
Total                    &  5756.2 &  13322.0 &   43.2 \\
\bottomrule
\end{tabular}
\]
\end{document}